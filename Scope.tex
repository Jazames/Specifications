\section{SCOPE}

\begin{slshape}
\color{blue} 
This section is a very top level, simple verbal description of what the item is and where it is used.   See the amplifier example under the \textbf{General} heading.
\bigskip
\end{slshape}


\begin{enumerate}[(a)]
	\item \textbf{\underline{General:}} \textbf{This document describes the design and verification requirements for the USU Controls Laboratory power amplifier module.  The module is used to provide current gain to the control voltages that drive the laboratory motors.}
\bigskip

\begin{slshape}
\color{blue}
	For a simple module such as the power amplifier module the above description is adequate. 
\end{slshape}
\bigskip


	\item \textbf{\underline{Acronyms:}}\begin{slshape} \color{blue}{I have not chosen to define acronyms because I have a thing about the overuse of acronyms.  You may not be so burdened and find that typing out a name is simply too tedious.  Put those acronyms here.}\end{slshape}
	\item \begin{slshape} \color{blue}{Additional short descriptive paragraphs can be added only if 
	needed for special classification, designation of alternate versions or 
	other material that is part of a top-level description.}\end{slshape}
\end{enumerate}

\begin{slshape}
\color{blue}
\StopSign  Take time to write this section for your project.  Being able to write a simple description of even complex things is a good indicator of how well you understand what you are planning.
\end{slshape}