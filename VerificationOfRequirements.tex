\section{VERIFICATION OF REQUIREMENTS}




\begin{slshape}
\color{blue}

We have to know that we designed what we were required to design by this document.  This process is called `verification' and it answers the fundamental question: ``Did we build what we said we would build?".  We verify requirements by testing to see if the requirements are met.  Design teams must test every requirement in order to prove that the requirement is met.  Testing each requirement means that each requirement must cast in some quantifiable way.  In this section you specify how you will test each requirement to show that it has been met.
\bigskip

Don't worry!  It is not as bad as it sounds.  Sometimes (not always!) you can verify or test a requirement simply by looking at the completed system to make sure some required thing is present. 
\bigskip

Testing often `takes it on the chin' in terms of project schedule.  Since integrated system testing typically occurs near the end of a project, the time for testing is compressed against the deadline.  People start short-cutting tests to stay on schedule.  Sometimes you may get away with it but it is never a good idea either technically or ethically.  Epic failures have occurred because of truncated testing.  One such failure occurred during the testing of the Hubble Space Telescope.  The following is an excerpt from the the official report detailing the failure.

\begin{quote}
Reliance on a single test method was a process which was clearly
vulnerable to simple error. Such errors had been seen in other telescope
programs, yet no independent tests were planned, although some simple tests to
protect against major error were considered and rejected. During the critical time
period, there was great concern about cost and schedule, which further inhibited
consideration of independent tests.\\
\bigskip
The Hubble Space Telescope Optical Systems Failure Report-NASA November 1990 
\end{quote}
\bigskip

If you are interested the whole report is available at (\url{https://www.ssl.berkeley.edu/~mlampton/AllenReportHST.pdf}).
\bigskip

The Hubble error wasn't caught until the telescope was deployed in space.  Can you imagine the cost of fixing this problem?  It is not simply a case of bundling you off with your instruments and putting you up in a fancy hotel for a week or two.  Some estimates set the price at about \$1 billion.
\bigskip

The Dilbert comic strip has a similar, and darkly amusing, view of testing truncation.
\bigskip  

(\url{http://dilbert.com/strip/2010-08-21})
\bigskip

(\url{http://dilbert.com/strip/2009-07-01})
\bigskip



	
The key to completing this section is that every requirement has an associated test.  The best practice in this section is to match the sub-paragraph numbers in the previous section to the sub-paragraph numbers in this section, e.g the requirement in 4.3.1.6 is covered by the test described in 5.3.1.6.
\bigskip

	Possible verification methods include:
	\bigskip
	
	\begin{enumerate}
		\item Inspection:\\

	Inspection is a method of verification consisting of investigation, 
	without the use of special laboratory appliances or procedures, to 
	determine compliance with requirements. Inspection is generally 
	nondestructive and includes (but is not limited to) visual examination, 
	manipulation, gauging, and measurement.

		\item Demonstration:\\

	Demonstration is a method of verification that is limited to readily 
	observable functional operation to determine compliance with 
	requirements. This method shall not require the use of special equipment 
	or sophisticated instrumentation.
	
		\item Analysis:\\

	Analysis is a method of verification, taking the form of the processing of 
	accumulated results and conclusions, intended to provide proof that 
	verification of a requirement has been accomplished. The analytical 
	results may be based on engineering study, compilation or interpretation 
	of existing information, similarity to previously verified requirements, 
	or derived from lower level examinations, tests, demonstrations, or 
	analyses.


		\item Direct Test:

	Test is a method of verification that employs technical means, including (but not 
	limited to) the evaluation of functional characteristics by use of special equipment
	or instrumentation, simulation techniques, and the application of established 
	principles and procedures to determine compliance with requirements.
			
	\end{enumerate}		
	
\end{slshape}

\subsection{Verify Coverage of Stakeholder Requirements}

\begin{slshape}
\color{blue}
The tester verifies that everything that the stakeholders have asked for are covered by one or more requirements.  It is a good idea for the requirements author(s) to perform a similar check at this point.  The tester is likely to do his own analysis or disagree on points in yours, but the exercise itself is valuable. And if you do the analysis you might as well write it down here.   
\end{slshape}


\subsection{Interface}

\subsubsection{Functional Interface Constraints}

\subsubsection{Functional Interface Requirements}

\subsubsection{Support Interface Constraints}

\subsubsection{Support Interface Requirements}

\subsection{Functional Requirements}

\subsubsection{Functional Method Constraints}

\subsubsection{Functional Design Requirements}

\subsection{Support Requirements}

\subsubsection{Support Method Constraints}

\subsubsection{Support Requirements} 


\begin{slshape}
	\color{blue}
  A tabulation of all the requirements and the testing method with a blank space for results is useful for whomever is doing the testing.
\end{slshape}

\begin{table}[h]
\centering
\begin{tabular}{|c|c|C{6cm}|c|c|}
\hline
\textbf{Paragraph Number} & \textbf{Test Type}& 
\textbf{Tester's Name} & \textbf{Pass/Fail} & \textbf{Date} \\
\hline
 & & & & \\
\hline
 & & & & \\
\hline
 & & & & \\
\hline
 & & & & \\
\hline
 & & & & \\
\hline
 & & & & \\
\hline
 & & & & \\
\hline
 & & & & \\
\hline
 & & & & \\
\hline
 & & & & \\
\hline
\end{tabular}
\end{table}

\begin{slshape}
\color{blue}
\StopSign Now read over your completed specification and make additions and corrections.  Find others who will be willing to read and comment on the specification (hopefully they will still like you when they are done).  The more eyes the better.  Ask yourself if you handed this spec to a competent classmate what would they build?
\bigskip

Congratulations!  You have written an engineering specification and that is no mean feat.
\end{slshape}